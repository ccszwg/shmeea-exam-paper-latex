% !TEX program = xelatex
\documentclass[10pt,answers,addpoints]{exam}
%\usepackage{exam}
\usepackage[fontset=shmeea]{ctex}
\usepackage{multicol}
\usepackage{tikz}
\usetikzlibrary{calc}
\usepackage{tasks}
\usepackage{gensymb}
\usepackage{xifthen}
\usepackage{newtxtext}
\usepackage[uprightGreek]{newtxmath}
\DeclareMathSizes{10}{12}{6.96}{5.04} 
\DeclareMathOperator{\iu}{i}
\usepackage[b5paper, top=2cm, bottom=3.5cm, inner=1.9cm, outer=1.8cm]{geometry}
%\usepackage[paperwidth=18.4cm, paperheight=26cm, top=2cm, bottom=3.5cm, hmargin=2cm]{geometry}
\usepackage[inline, shortlabels]{enumitem}
%\usepackage[subscriptcorrection,slantedGreek,nofontinfo]{mtpro2}
\setlength\fillinlinelength{1.5cm}
\newcommand{\abs}[1]{
    \ensuremath{\left\lvert#1\right\rvert}
}
\newcounter{numpofq}

\newcommand{\setcountertopartsofq}[1]{%
  \def\qnotemp{#1}%
  \setcounter{numpofq}{1}%
  \numpofqrelay
}%

\def\numpofqrelay{%
  \expandafter\ifx\csname Pg@part@\qnotemp
                  @\arabic{numpofq}\endcsname\relax
    % This part number doesn't exist; back up one and exit:
    \addtocounter{numpofq}{-1}%
    \let\nextnumpofqrelay=\relax
  \else
    % This part number exists; try the next number:
    \addtocounter{numpofq}{1}%
    \let\nextnumpofqrelay=\numpofqrelay
  \fi
  \nextnumpofqrelay
}% numpofqrelay

\makeatletter
\long\def\questionvbox#1{%
  \par\smallskip
  \vbox{%
    \parindent=\leftmargin
    \leftskip=\@totalleftmargin
    \advance\leftskip-\leftmargin
    \advance\@totalleftmargin-\leftmargin
    \advance\linewidth\leftmargin
    #1%
  }%
  \nobreak
}
\def\wordnum#1{\expandafter\word\csname c@#1\endcsname}
\def\word#1{%
  \ifcase #1 zero\or one\or two\or three\or four\or five\or six\or
  seven\or eight\or nine\or ten\or eleven\or twelve\or thirteen\or
  fourteen\or fifteen\or sixteen\or seventeen\or eighteen\or
  nineteen\or twenty\else too many\fi
}% word
\makeatother
%\DeclareMathSizes{10.95}{10}{7}{7}
\makeatletter
%\DeclareMathSizes{11}{19}{13}{9}
\newcommand\ExamYear[1]{\renewcommand\@ExamYear{#1}}
\newcommand\@ExamYear{\the\year{}}
\newcommand\ExamSubject[1]{\renewcommand\@ExamSubject{#1}}
\newcommand\@ExamSubject{}
\newcommand{\ShmeeaTitle}{
    \begin{center}
        {\Large{\@ExamYear}年普通高等学校招生全国统一考试\par}
        \vspace*{1\baselineskip}
        {\huge\bfseries{上海\quad{\@ExamSubject}试卷}}
        \vspace*{1\baselineskip}
    \end{center}
}
\newcommand{\ShmeeaNotice}[2]{
    {
        \heiti 考生注意
        \begin{enumerate}[leftmargin=34pt, nosep, topsep=0pt]
            \item 本场考试时间 #1 分钟, 试卷共 \numpages 页, 满分 \numpoints 分, 答题纸共 #2 页.
            \item 作答前, 在答题纸正面填写姓名、准考证号,反面填写姓名,将核对后的条形码贴在答题纸指定位置.
            \item 所有作答务必填涂或书写在答题纸上与试卷题号对应的区域,不得错位,在试卷上作答一律不得分.
            \item 用 2B 铅笔作答选择题,用黑色字迹钢笔、水笔或圆珠笔作答非选择题.
        \end{enumerate}
    }
}
\firstpagefooter{{\windowsong 上海市教育考试院}{\quad}{\bfseries 保留版权}}{}{\windowsong {\@ExamYear}年高考{\@ExamSubject} 第 \thepage 页(共 \numpages 页)}
\runningfooter{\oddeven{}{\windowsong {\@ExamYear}年高考{\@ExamSubject} 第 \thepage 页(共 \numpages 页)}}{}{\oddeven{\windowsong {\@ExamYear}年高考{\@ExamSubject} 第 \thepage 页(共 \numpages 页)}{}}
\makeatother
\author{T}
\newcounter{bigpartcounter}
\newcommand{\bigpart}[4]{
    \stepcounter{bigpartcounter}
    \uplevel{\bfseries{\chinese{bigpartcounter}、#2\enspace{(本大题共有 \numqinrange{#1} 题,满分 {\pointsinrange{#1}} 分 {#3})}\enspace{{#4}}}}
}
\newcommand{\detail}[1]{
    %\ifnum#1=1
    %\ifthenelse{{\the\numqinrange{#1}}=1}
    %{第 \firstqinrange{#1} 题 \pointsofquestion{\the\firstqinrange{#1}} 分}
    {第 \firstqinrange{#1}\~\lastqinrange{#1} 题 \firstqinrange{#1} 分}
    %\else
    %第 \firstqinrange{#1}\~\lastqinrange{#1} 题每题 \pointsofquestion{\firstqinrange{#1}} 分
    %\fi
}
\newcommand{\solvingtitle}[1]{%,共 \arabic{numpofq} 部分
    \setcountertopartsofq{\arabic{question}}{\bfseries (本题满分 \pointsofquestion{\thequestion} 分{#1})\par}
}
\newcommand{\bracketfillin}[1][]{
    %\ifthenelse{\isempty{#1}}{
    %}{
    %}
    \ifprintanswers
        \nolinebreak\dotfill\mbox{(\fillin[#1][0.6cm])}
    \else
        \nolinebreak\dotfill\mbox{(\hspace{0.6cm})}
    \fi
}
%\newcommand{\bracketfillin}[1][1]{\nolinebreak\dotfill\mbox{(\hspace{0.8cm})} }
%\newcommand{\bracketfillin}[1][1]{\nolinebreak\dotfill\mbox{\raisebox{-1.8pt}(\hspace{1cm})} }
\newlength\labelwd
\settowidth\labelwd{(A)}
\settasks{counter-format=(tsk[A]), label-width=\labelwd, after-item-skip=0pt}
\newcommand{\fourch}[4]{
\begin{tasks}(4)
    \task #1
    \task #2
    \task #3
    \task #4
\end{tasks}
}

\newcommand{\twoch}[4]{
\begin{tasks}(2)
    \task #1
    \task #2
    \task #3
    \task #4
\end{tasks}
}

\newcommand{\onech}[4]{
\begin{tasks}(1)
    \task #1
    \task #2
    \task #3
    \task #4
\end{tasks}
}

\newlength\widthcha
\newlength\widthchb
\newlength\widthchc
\newlength\widthchd
\newlength\widthch
\newlength\tabmaxwidth

\setlength\tabmaxwidth{0.96\textwidth}
\newlength\fourthtabwidth
\setlength\fourthtabwidth{0.25\textwidth}
\newlength\halftabwidth
\setlength\halftabwidth{0.5\textwidth}

\newcommand{\choice}[4]{%
\settowidth\widthcha{ (A) .#1}\setlength{\widthch}{\widthcha}%
\settowidth\widthchb{ (B) .#2}%
\ifdim\widthch<\widthchb\relax\setlength{\widthch}{\widthchb}\fi%
\settowidth\widthchc{ (C) .#3}%
\ifdim\widthch<\widthchc\relax\setlength{\widthch}{\widthchc}\fi%
\settowidth\widthchd{ (D) .#4}%
\ifdim\widthch<\widthchd\relax\setlength{\widthch}{\widthchd}\fi%
\ifdim\widthch<\fourthtabwidth
\fourch{#1}{#2}{#3}{#4}
\else\ifdim\widthch<\halftabwidth
    \twoch{#1}{#2}{#3}{#4}
\else
    \onech{#1}{#2}{#3}{#4}
\fi
\fi
}
\ExamYear{2018}
\ExamSubject{数学}
\everymath{\displaystyle}
%\pointsinmargin
\pointname{分}
\renewcommand{\solutiontitle}{\textbf{解}\enspace}
\renewcommand{\thepartno}{\arabic{partno}}
\renewcommand{\partshook}{\setlength{\parindent}{\leftmargin}\setlength{\topsep}{0pt}\setlength{\itemsep}{0pt}}
\addpoints
%\bracketedpoints
\boxedpoints
\pointsinleftmargin
%\pointsdroppedatright
\begin{document}
    \ShmeeaTitle
    \ShmeeaNotice{120}{2}

    \begin{questions}
        \bigpart{blanks}{填空题}{,第 \firstqinrange{blanks4}\textasciitilde{}\lastqinrange{blanks4} 题每题 4 分,第 \firstqinrange{blanks5}\textasciitilde{}\lastqinrange{blanks5} 题每题 5 分}
        {
            考生应在答题纸的相应位置直接填写结果.
        }
        \begingradingrange{blanks}
            \begingradingrange{blanks4}
            \question[4]
                行列式 $\begin{vmatrix}4&1\\2&5\end{vmatrix}$ 的值为 \fillin[$18$].
            \question[4]
                双曲线 $\frac{x^2}{4}-y^2=1$ 的渐近线方程为 \fillin[$y=\pm\frac12x$].
            \question[4]
                在 $(1+x)^7$ 的二项展开式中,$x^2$ 项的系数为 \fillin[$21$] (结果用数值表示).
            \question[4]
                设常数 $a\in\mathbf{R}$,函数 $f(x)=\log_2(x+a)$,若 $f(x)$ 的反函数的图像经过点 $(3,1)$,则 $a=$ \fillin[$7$].
            \question[4]
                已知复数 $z$ 满足 $(1+\iu)z=1-7\iu$($\iu$ 是虚数单位),则 $\abs{z}=$ \fillin[$5$].
            \question[4]
                记等差数列$\left\{a_n\right\}$的前 $n$ 项和为 $S_n$. 若 $a_3=0$,$a_6+a_7=14$,则 $S_7=$ \fillin[$14$].
            \endgradingrange{blanks4}
            \begingradingrange{blanks5}
            \question[5]
                已知 $\alpha\in\,\left\{\,-2,-1,-\frac12,\frac12,1,2,3\,\right\}\,$. 若幂函数 $f(x) = {x^\alpha}$ 为奇函数,且在 $(0,+\infty)$ 上递减,则 $\alpha=$ \fillin[$-1$].
            \question[5]
                在平面直角坐标系中,已知点 $A(-1,0)\,$、$\,B(2,0)$,$E\,$、$\,F$ 是 $y$ 轴上的两个动点,且 $\abs{\overrightarrow{EF}}=2$,则$\overrightarrow{AE}\cdot\overrightarrow{BF}$的最小值为 \fillin[$-3$].
            \question[5]
                有编号互不相同的五个砝码,其中 $5$ 克、$3$ 克、$1$ 克砝码各一个,$2$ 克砝码两个. 从中随机选取三个,则这三个砝码的总质量为 $9$ 克的概率是 \fillin[$\frac15$] (结果用最简分数表示).
            \question[5]
                设等比数列 $\{a_n\}$ 的通项公式为 $a_n=q^{n+1}\enspace(n\in\mathbf{N}^*)$,前 $n$ 项和为 $S_n$. 若$\lim_{n\to\infty}\frac{S_n}{a_{n+1}}=\frac12$,则 $q=$ \fillin[$3$].
            \question[5]
                已知常数 $a>0$,函数 $f(x)=\frac{2^x}{2^x+ax}$ 的图像经过点 $P\,\left(p,\frac65\right)\,$、$Q\,\left(q,-\frac15\right)\,$,若 $2^{p + q} = 36pq$,则 $a=$ \fillin[$6$].
            \question[5]
                已知实数 $x_1$、$x_2$、$y_1$、$y_2$ 满足: $x_1^2+y_1^2=1$,$x_2^2+y_2^2=1$,$x_1x_2+y_1y_2=\frac12$,则 $\frac{\abs{x_1+y_1-1}}{\sqrt2}+\frac{\abs{x_2+y_2-1}}{\sqrt2}$ 的最大值为 \fillin[$\sqrt2+\sqrt3$].
            \endgradingrange{blanks5}
        \endgradingrange{blanks}
        \bigpart{choices}{选择题}{,每题 5 分}
        {
            每题有且只有一个正确选项,考生应在答题纸的相应位置,将代表正确选项的小方格涂黑.
        }
        \begingradingrange{choices}
            \question[5] 设 $P$ 是椭圆 $\frac{x^2}{5}+\frac{y^2}{3}=1$ 上的动点,则 $P$ 到该椭圆的两个焦点的距离之和为 \bracketfillin[C].
            \choice{$2\sqrt2$}{$2\sqrt3$}{$2\sqrt5$}{$4\sqrt2$}
            \question[5] 已知 $a\in\mathbf{R}$,则“$a>1$”是“$\frac1a<1$” 的 \bracketfillin[A].
            \choice{充分非必要条件}{必要非充分条件}{充要条件}{既非充分又非必要条件}
            \question[5]
            \noindent
            \begin{minipage}[t]{0.65\textwidth}
                \fussy
                《九章算术》中,称底面为矩形而有一侧棱垂直于底面的四棱锥为阳马. 设 $AA_1$ 是正六棱柱的一条侧棱,如图. 
                若阳马以该正六棱柱的顶点为顶点,以 $AA_1$ 为底面矩形的一边,则这样的阳马的个数是 \bracketfillin[D].
                \choice{$4$}{$8$}{$12$}{$16$}
            \end{minipage}\hfill
            \begin{minipage}[t]{0.3\textwidth}
                \centering\raisebox{\dimexpr \topskip-\height}{%
                \begin{tikzpicture}[smooth, thick, scale=0.55]
                    \coordinate (A) at (0,0);
                    \coordinate (B) at (1.73,1);
                    \coordinate (C) at (1,1.73);
                    \coordinate (D) at (-2.46,1.73);
                    \coordinate (E) at (-4.19,0.73);
                    \coordinate (F) at (-3.46,0);
                    \coordinate (A1) at (0,6);
                    \coordinate (B1) at (1.73,7);
                    \coordinate (C1) at (1,7.73);
                    \coordinate (D1) at (-2.46,7.73);
                    \coordinate (E1) at (-4.19,6.73);
                    \coordinate (F1) at (-3.46,6);
                    \draw (A) node[below] {$A$};
                    \draw (A1) node[above] {$A_1$};
                    \begin{scope}[densely dashed]
                        \draw (B) -- (C) -- (D) -- (E);
                        %\draw (E1) -- (F1) -- (A1) -- (B1) -- (C1) -- (D1) -- cycle;
                        \draw (C) -- (C1);
                        \draw (D) -- (D1);
                    \end{scope}
                    \draw (E) -- (F) -- (A) -- (B);
                    \draw (E1) -- (F1) -- (A1) -- (B1) -- (C1) -- (D1) -- cycle;
                    \draw (E) -- (E1);
                    \draw (F) -- (F1);
                    \draw (A) -- (A1);
                    \draw (B) -- (B1);
                \end{tikzpicture}
                }
            \end{minipage}
            \question[5] 设 $D$ 是含数 $1$ 的有限实数集,$f(x)$ 是定义在 $D$ 上的函数. 若 $f(x)$ 的图像绕原点逆时针旋转 $\frac\pi6$ 后与原图像重合,则在以下各项中,$f(1)$ 的可能取值只能是 \bracketfillin[B].
            \choice{$\sqrt3$}{$\frac{\sqrt3}{2}$}{$\frac{\sqrt3}{3}$}{$0$}
        \endgradingrange{choices}
        \bigpart{solving}{解答题}{}
        {
            解答下列各题必须在答题纸的相应位置写出必要的步骤.
        }
        \begingradingrange{solving}
            \question \solvingtitle{}%\detail{\thequestion}
            \noindent
            \begin{minipage}[t]{0.65\textwidth}
                \fussy
                \questionvbox{
                已知圆锥的顶点为 $P$,底面圆心为 $O$,半径为 $2$.
                }
                    \begin{parts}
                        \part[6] 设圆锥的母线长为 $4$, 求圆锥的体积;
                        \part[8] 设 $PO=4$,$OA$、$OB$ 是底面半径,且 $\angle{AOB}=90\degree$,$M$ 为线段 $AB$ 的中点,如图. 求异面直线 $PM$ 与 $OB$ 所成的角的大小.
                    \end{parts}
            \end{minipage}\hfill
            \begin{minipage}[t]{0.3\textwidth}
                \centering\raisebox{\dimexpr \topskip-\height}{%
                \begin{tikzpicture}[smooth, thick, scale=0.45]

                    \newcommand{\radiusx}{3}
                    \newcommand{\radiusy}{1.2}
                    \newcommand{\height}{6}

                    \coordinate (O) at (0,0);
                    \coordinate (P) at (0,\height);
                    \coordinate (lp) at (-{\radiusx*sqrt(1-(\radiusy/\height)*(\radiusy/\height))},{\radiusy*(\radiusy/\height)});
                    \coordinate (rp) at ({\radiusx*sqrt(1-(\radiusy/\height)*(\radiusy/\height))},{\radiusy*(\radiusy/\height)});
                    \coordinate (A) at (-1,-1.13);
                    \coordinate (B) at (\radiusx,0);
                    \coordinate (M) at ($.5*(A)+.5*(B)$);
                    %\draw[fill=gray!30] (a)--(0,\height)--(b)--cycle;
                    \draw (lp)--(0,\height)--(rp);%--cycle;

                    %\fill[gray!50] circle (\radiusx{} and \radiusy);
                    %\draw circle (\radiusx{} and \radiusy);

                    \begin{scope}
                    \clip ([xshift=-2mm]lp) rectangle ($(rp)+(1mm,-2*\radiusy)$);
                    \draw circle (\radiusx{} and \radiusy);
                    \end{scope}
                    
                    \begin{scope}
                    \clip ([xshift=-2mm]lp) rectangle ($(rp)+(1mm,2*\radiusy)$);
                    \draw[densely dashed] circle (\radiusx{} and \radiusy);
                    \end{scope}
                    
                    %\draw[dashed] (0,\height)|-(\radiusx,0) node[right, pos=.25]{$h$} node[above,pos=.75]{$r$};
                    \draw[dashed] (0,\height)|-(\radiusx,0);
                    
                    \draw (P) node[left] {\small $P$};
                    \draw (O) node[left] {\small $O$};
                    \draw (A) node[below] {\small $A$};
                    \draw (B) node[right] {\small $B$};
                    \draw (M) node[right=2pt,below=-3pt] {\small $M$};
                    \begin{scope}[densely dashed]
                        \draw (B) -- (A);
                        \draw (P) -- (A);
                        \draw (P) -- (M);
                    \end{scope}
                    %\draw (0,.15)-|(.15,0);
                \end{tikzpicture}
                }
            \end{minipage}
                \begin{solutionordottedlines}[2cm]
                    \begin{small}
                        \begin{enumerate*}[(1),nosep,topsep=0pt]
                            \item $V=\frac83\sqrt3\pi$;
                            \item 异面直线 $PM$ 与 $OB$ 所成的角的大小为 $\arccos\frac{\sqrt2}{6}$.
                        \end{enumerate*}
                    \end{small}
                \end{solutionordottedlines}
            \question \solvingtitle{}%\detail{\thequestion}
            \questionvbox{
                设常数 $a\in\mathbf{R}$,函数 $f(x)=a\sin 2x+2\cos^2 x$
            }
                \begin{parts}
                        \part[6] 若 $f(x)$ 为偶函数,求 $a$ 的值;
                        \part[8] 若 $f\left(\frac\pi4\right)=\sqrt3+1$,求方程 $f(x)=1-\sqrt2$ 在区间 $[-\pi,\pi]$ 上的解.
                \end{parts}
                \begin{solutionordottedlines}[2cm]
                    \begin{small}
                        \begin{enumerate*}[(1),nosep,topsep=0pt]
                            \item $a=0$;
                            \item 零点为 $x=-\frac{13}{24}\pi,-\frac{11}{24}\pi,-\frac{5}{24}\pi,\frac{19}{24}\pi$.
                        \end{enumerate*}
                    \end{small}
                \end{solutionordottedlines}
            \question \solvingtitle{}
            \questionvbox{%
            某群体的人均通勤时间,是指单日内该群体中成员从居住地到工作地的平均用时,某地上班族 $S$ 中的成员仅以自驾或公交方式通勤. 分析显示:当 $S$ 中 $x\%\enspace(0<x<100)$ 的成员自驾时,自驾群体的人均通勤时间为
                $$
                    f(x)=
                    \begin{cases}
                    30,&0<x\leq 30,\\
                    2x+\frac{1800}x-90,&30<x<100    
                    \end{cases}
                    \text{(单位:分钟),}
                $$
                而公交群体的人均通勤时间不受 $x$ 影响,恒为 $40$ 分钟. 试根据上述分析结果回答下列问题:
            }
                \begin{parts}
                    \part[6] 当 $x$ 在什么范围内时,公交群体的人均通勤时间少于自驾群体的人均通勤时间?
                    \part[8] 求该地上班族 $S$ 的人均通勤时间 $g(x)$ 的表达式;讨论 $g(x)$ 的单调性,并说明其实际意义.
                \end{parts}
                \begin{solutionordottedlines}[2cm]
                    \begin{small}
                        \begin{enumerate*}[(1),nosep,topsep=0pt]
                            \item $x\in(45,100)$时,公交群体的人均通勤时间少于自驾群体的人均通勤时间;\\
                            \item $g(x)=\begin{cases}40-\frac{x}{10},&0<x\leq 30,\\\frac1{50}(x-32.5)^2+36.875,&30<x<100.\end{cases}$
                            \\
                            $x\in(0,32.5]$ 时,$g(x)$ 单调递减;$x\in(32.5,100)$ 时,$g(x)$ 单调递增.
                            \\
                            实际意义是当有 $32.5\%$ 的上班族采用自驾方式时,上班族整体的人均通勤时间最短. 适当的增加自驾比例,可以充分地利用道路交通,实现整体效率提升;但自驾人数过多,则容易导致交通拥堵,使得整体效率下降.
                        \end{enumerate*}
                    \end{small}
                \end{solutionordottedlines}
            \question \solvingtitle{}
            \questionvbox{
                设常数 $t>2$,在平面直角坐标系 $xOy$ 中,已知点 $F(2,0)$,直线 $l:x=t$,曲线 $\Gamma:y^2=8x\enspace(0\leq x\leq t,y\geq 0)$. $l$ 与 $x$ 轴交于点 $A$、与 $\Gamma$ 交于点 $B$. $P$、$Q$ 分别是曲线 $\Gamma$ 与线段 $AB$ 上的动点.
            }
            \begin{parts}
                \part[4] 用 $t$ 为表示点 $B$ 到点 $F$ 的距离;
                \part[6] 设 $t\!=\!3$,$\abs{FQ}=2$ ,线段 $OQ$ 的中点在直线 $FP$ 上,求 $\triangle AQP$ 的面积;
                \part[6] 设 $t\!=\!8$,是否存在以 $FP$、$FQ$为邻边的矩形 $FPEQ$,使得点 $E$ 在 $\Gamma$ 上?\mbox{若存在},求点 $P$ 的坐标;若不存在,说明理由.
            \end{parts}
            \begin{solutionordottedlines}[2cm]
                \begin{small}
                    \begin{enumerate*}[(1),nosep,topsep=0pt]
                        \item $\abs{BF}=t+2$;%点 $B$ 到点 $F$ 的距离
                        \item $S_{\triangle AQP}=\frac76\sqrt3$;%$\triangle AQP$ 的面积 
                        \item 存在,$P\left(\frac25,\frac45\sqrt5\right)$.
                    \end{enumerate*}
                \end{small}
            \end{solutionordottedlines}
            \question \solvingtitle{}
            \questionvbox{
                给定无穷数列 $\{a_n\}$,若无穷数列 $\{b_n\}$ 满足:对任意 $n\in\mathbf{N}^*$,都有 $\abs{b_n-a_n}\leq 1$ ,则称 $\{b_n\}$ 与 $\{a_n\}$“接近”.
            }
            \begin{parts}
                \part[4] 设 $\{a_n\}$ 是首项为 $1$,公比为 $\frac12$ 的等比数列,$b_n=a_{n+1}+1\enspace,n\in\mathbf{N}^*$,判断数列 $\{b_n\}$ 是否与 $\{a_n\}$ 接近,并说明理由;
                \part[6] 设数列 $\{a_n\}$ 的前四项为:$a_1=1$,$a_2=2$,$a_3=4$,$a_4=8$,$\{b_n\}$ 是一个与 $\{a_n\}$ 接近的数列,记集合 $M=\{x|x=b_i,i=1,2,3,4\}$,求 $M$ 中元素的个数 $m$;
                \part[8] $\{a_n\}$ 是公差为 $d$ 的等差数列. 若存在数列 $\{b_n\}$ 满足:$\{b_n\}$ 与 $\{a_n\}$ 接近,且在 $b_2-b_1,b_3-b_2,\cdots,b_{201}-b_{200}$ 中至少有 $100$ 个为正数,求 $d$ 的取值范围.
            \end{parts}
            \begin{solutionordottedlines}[2cm]
                \begin{small}
                    \begin{enumerate*}[(1),nosep,topsep=0pt]
                        \item 接近;
                        \item $m=3\text{\ 或\ } 4$;
                        \item $d\in(-2,+\infty)$.
                    \end{enumerate*}
                \end{small}
            \end{solutionordottedlines}
        \endgradingrange{solving}
    \end{questions}
\end{document}
